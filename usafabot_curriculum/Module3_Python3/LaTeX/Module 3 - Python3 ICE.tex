\documentclass{handout}


\SetCourseTitle{ECE495: Fundamentals of Robotics Research}
\SetHandoutTitle{Module 3 - Python3 for Robotics}
%\SetDueDate{8 Sep at 0715 (via Gradescope)}
%\ShowAllBlanks

%\showsoln \setsolncolor{red}

\newlist{todolist}{itemize}{2}
\setlist[todolist]{label=$\square$}
\usepackage{pifont}
\newcommand{\cmark}{\ding{51}}%
\newcommand{\xmark}{\ding{55}}%
\newcommand{\done}{\rlap{$\square$}{\raisebox{2pt}{\large\hspace{1pt}\cmark}}%
	\hspace{-2.5pt}}
\newcommand{\wontfix}{\rlap{$\square$}{\large\hspace{1pt}\xmark}}

\usepackage{hyperref}

\definecolor{code}{HTML}{ECF8F4}
\definecolor{comments}{HTML}{5269A5}

\usepackage[T1]{fontenc}
\lstset{%
	language=bash, upquote=true,
	otherkeywords={rostopic, rosnode, rosrun, roscore, cd, ls, sudo, nano, echo, mkdir, touch, chmod, catkin\_make, rosmsg, rosservice, catkin\_create\_pkg, rospack, ssh, rosed},
	showspaces=false, showtabs=false, showstringspaces=false, upquote=true, tabsize=4,
	literate={~} {$\sim$}{1},
	showstringspaces=false,
	xleftmargin=0.06\textwidth,
	linewidth=0.99\textwidth,
	columns=fullflexible,
	backgroundcolor=\color{code},
	keepspaces=true,
	breaklines=true,
	basicstyle={\small\fontfamily{fvm}\fontseries{m}\selectfont},
	keywordstyle={\small\fontfamily{fvm}\fontseries{b}\selectfont},
	commentstyle={\color{comments}\small\fontfamily{fvm}\itshape\selectfont},
	belowcaptionskip=10pt,
	float=h
}

\graphicspath{{./figs/}}

\begin{document}
\maketitle

\begin{figure}[H]
	\centering
	\includegraphics[width=.75\textwidth]{Cover.PNG}
\end{figure}

\textbf{Lesson Objectives:}
\begin{enumerate} \setlength\itemsep{0em}
	\item Use Object Oriented Programming (OOP) to develop advanced chat client
\end{enumerate}

\textbf{Agenda:}
\begin{enumerate} \setlength\itemsep{0em}
	\item In Class Exercise.
\end{enumerate}

\newpage
\clearpage
\pagebreak

\section{In Class Exercise.}

\begin{enumerate}
	\item On the master, create a package called \texttt{ice\_python}.
	
\begin{lstlisting}[language=bash]
cd ~/master_ws/src/ece495_fall2021-USERNAME/master/
catkin_create_pkg ice_python rospy std_msgs
cd ~/master_ws
catkin_make
source ~/.bashrc
\end{lstlisting}

	\item Create two chat files
\begin{lstlisting}[language=bash]
cd ~/master_ws/src/ece495_fall2021-USERNAME/master/ice_python/src
touch chat_client.py
touch chat_server.py
chmod +x *.py # makes all python files executable
\end{lstlisting}
	\item Edit the chat\_client.py (double click in file browser)
	\item Add necessary include statements
\begin{lstlisting}[language=python]
#!/usr/bin/env python3
import rospy
from std_msgs.msg import String
\end{lstlisting}
	\item Create a Client class with a constant class dictionary that is used to map numbers to messages. For example:
\begin{lstlisting}[language=python]
class Client:
    MESSAGE = {1: "Hello!", 2: "How are you?", 3: "Where are you from?", 
        4: What are you doing today?"}
\end{lstlisting}
	\item Initialize a class variable to store the message and the publisher to publish the message to the "client" topic.
\begin{lstlisting}[language=python]
def __init__(self):
    self.message = String()
    self.pub = rospy.Publisher('client', String, queue_size=10)
\end{lstlisting}
	\item Initialize the subscriber that will receive String messages over the "server" topic. When the message is received, it should call the callback function "callback\_received".
\begin{lstlisting}[language=python]
rospy.Subscriber('server', String, self.callback_received)
\end{lstlisting}
	\item Initialize a timer that calls a function "callback\_input" every second.
\begin{lstlisting}[language=python]
rospy.Timer(rospy.Duration(1), self.callback_input)
\end{lstlisting}
	\item Create the "callback\_input" class function that has two parameters: the class and an TimerEvent object. The TimerEvent object provides timing information for the callback. This website has more information on the ROS Timer \url{http://wiki.ros.org/rospy/Overview/Time#Timer}
	\begin{enumerate}
		\item This function should tell the user to input a number that corresponds to a message.
		\item The function should check the user input to ensure that it is a valid number and if not, then continue to ask the user for input
	\end{enumerate}
\begin{lstlisting}[language=python]
def callback_input(self, event):
    print("Using the number keys, input a value that corresponds to one 
        of the messages:", self.MESSAGE)
    valid = False
    while not valid:
        chat_str = input()
        try:
           val = int(chat_str)
           if(0 < val < 5):
               self.message = self.MESSAGE[val]
               self.pub.publish(self.message)
               valid = True
           else:
               print("This is not a valid input, please input a number
                   that corresponds to the messages:", self.MESSAGE)
       except ValueError:
           print("This is not a valid input, please input a number that 
               corresponds to the messages:", self.MESSAGE)
\end{lstlisting}

	\item Create the "callback\_received" class function that is called when the client receives a response from the server, this function will have two parameters: the class and the data received. The function should print both the message sent and the message received.
	
\begin{lstlisting}[language=python]
def callback_received(self, data):
    print("Client sent", self.message)
    print("Server responded", data.data)
\end{lstlisting}

	\item Create the main function that initializes the client node and class and runs forever.
\begin{lstlisting}[language=python]
if __name__ == "__main__":
    rospy.init_node('client', anonymous = True)
    
    c = Client()
    rospy.spin()
\end{lstlisting}
	\item At this point your client should be complete. Note that you have both a publisher and subscriber within one node. Using a Class allows us to facilitate this passing of information among functions.
	\item Answer the following questions:
	\begin{enumerate}
		\item Does anyone notice an issue with where we are requesting user input?
		\soln{.4in}{We are waiting for user input within a function that would like to be called every second. What will ROS do?}
		\item Could there be any timing issues with our request for user input and the response from the server?
		\soln{.5in}{Yes, as we will see, the print asking for user input will often be displayed before we get a response from the server}
		\item What may be a better method to implement a chat client like this other than subscriber/publisher?
		\soln{.5in}{Using Services.}
	\end{enumerate}
	\item Edit the chat\_server.py file (double click in file browser)
	\item Add necessary include statements
\begin{lstlisting}[language=python]
#!/usr/bin/env python3
import rospy
from std_msgs.msg import String
\end{lstlisting}
	\item Create a Server class with a constant dictionary used to map messages to responses. For example:
\begin{lstlisting}[language=python]
class Server:
    MESSAGE = {"Hello!": "Hi there!", 
        "How are you?": "I am great, thanks for asking!", 
        "Where are you from?": "I am from Las Vegas.", 
        "What are you doing today?": "ECE495!"}
    
    
    def __init__(self):
\end{lstlisting}
	\item Initialize a class variable to store the message and the publisher to publish the response to the "server" topic.
	\item Initialize the subscriber that will receive String messages over the "client" topic. When the message is received, it should call the callback function "callback\_received".
	\item Create the "callback\_received" class function that is called when the server receives a message from the client; this function will have two parameters: the class and the data received. The function should print the message from the client, the appropriate response from the dictionary, and then publish the response.
	\item Create the main function that initializes the server node and class and runs forever.
	\item At this point your server should be complete.
	\item Open a new terminal and run roscore.
	\item Open a new terminal and run the "chat\_client.py" node.
	\item Open a new terminal and run the "chat\_server.py" node.
	\item Test the operation of your chat bot. Ensure that it still works with invalid input.
\end{enumerate}

\textbf{Checkpoint. Take a screenshot or show the instructor the following:}
\begin{enumerate}
	\item List of running topics.
	\item The \textit{rqt\_graph} for the nodes and topics currently running.
	\item An echo of messages sent via the chat topic.
	\item Show what type of message is sent over the chat topic (Hint: use the rostopic command).
\end{enumerate}

\section{Assignments.}
	\begin{todolist}
		\item Finish Unit 5 and EXAM
	\end{todolist}

\section{Next time.}
	\begin{itemize}
		\item Module 4: Driving the Robot
	\end{itemize}

\end{document}
