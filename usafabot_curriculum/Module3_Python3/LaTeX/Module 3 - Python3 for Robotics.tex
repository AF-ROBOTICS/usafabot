\documentclass{handout}


\SetCourseTitle{ECE495: Fundamentals of Robotics Research}
\SetHandoutTitle{Module 3 - Python3 for Robotics}
%\SetDueDate{8 Sep at 0715 (via Gradescope)}
%\ShowAllBlanks

%\showsoln \setsolncolor{red}

\newlist{todolist}{itemize}{2}
\setlist[todolist]{label=$\square$}
\usepackage{pifont}
\newcommand{\cmark}{\ding{51}}%
\newcommand{\xmark}{\ding{55}}%
\newcommand{\done}{\rlap{$\square$}{\raisebox{2pt}{\large\hspace{1pt}\cmark}}%
	\hspace{-2.5pt}}
\newcommand{\wontfix}{\rlap{$\square$}{\large\hspace{1pt}\xmark}}

\usepackage{hyperref}

\definecolor{code}{HTML}{ECF8F4}
\definecolor{comments}{HTML}{5269A5}

\usepackage[T1]{fontenc}
\lstset{%
	language=bash, upquote=true,
	otherkeywords={rostopic, rosnode, rosrun, roscore, cd, ls, sudo, nano, echo, mkdir, touch, chmod, catkin\_make, rosmsg, rosservice, catkin\_create\_pkg, rospack, ssh, rosed},
	showspaces=false, showtabs=false, showstringspaces=false, upquote=true, tabsize=4,
	literate={~} {$\sim$}{1},
	showstringspaces=false,
	xleftmargin=0.06\textwidth,
	linewidth=0.99\textwidth,
	columns=fullflexible,
	backgroundcolor=\color{code},
	keepspaces=true,
	breaklines=true,
	basicstyle={\small\fontfamily{fvm}\fontseries{m}\selectfont},
	keywordstyle={\small\fontfamily{fvm}\fontseries{b}\selectfont},
	commentstyle={\color{comments}\small\fontfamily{fvm}\itshape\selectfont},
	belowcaptionskip=10pt,
	float=h
}

\graphicspath{{./figs/}}

\begin{document}
\maketitle

\begin{figure}[H]
	\centering
	\includegraphics[width=.75\textwidth]{Cover.PNG}
\end{figure}

\textbf{Lesson Objectives:}
\begin{enumerate} \setlength\itemsep{0em}
	\item Learn fundamental concepts of Python
	\item Learn basic syntax of Python
	\item Understand Object Oriented Programming
	\item Develop basic operational understanding of Python through application
\end{enumerate}

\textbf{Agenda:}
\begin{enumerate} \setlength\itemsep{0em}
	\item The Construct: Python3 for Robotics.
\end{enumerate}

\newpage
\clearpage
\pagebreak

\section{The Construct: Python3 for Robotics.}
In the next few lessons you will use the free online ROS education program called The Construct. We will now look at the Python3 programming language as it relates to robotics.

\begin{enumerate}
	\item Browse to \url{https://app.theconstructsim.com/} and login.
	\item In the left menu, select \textbf{Courses}.
	\item Within the \textit{Course Catalog}, filter by \textbf{Basic ROS}.
	\item Find the \textit{Python3 For Robotics} course and click \textbf{Start Learning}.
\end{enumerate}

\section{Assignments.}
	\begin{todolist}
		\item \textbf{Due beginning of Lesson 8} - Unit 1: Introduction \& Unit 2: Python Essentials
		\item \textbf{Due beginning of Lesson 9} - Unit 3: Conditional Statements \& Loops \& Unit 4: Functions 
		\item \textbf{Due beginning of Lesson 10} - Unit 5: Python Classes \& OOP \& EXAM - Ultimate Code Foundation Challenge
	\end{todolist}

\section{Next time.}
	\begin{itemize}
		\item \textbf{Lesson 8} - Unit 1 \& 2 quiz
		\item \textbf{Lesson 9} - Unit 3 \& 4 quiz
		\item \textbf{Lesson 10} - Module 4: Driving the robot using Teleop Twist Keyboard
	\end{itemize}

\end{document}
